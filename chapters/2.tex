\documentclass[output=paper]{langscibook}
\ChapterDOI{10.5281/zenodo.6762272}

\author{Gregory L. Thompson\affiliation{Brigham Young University}}
\title[Current trends in language assessment]
      {Current trends in language assessment: Using alternative assessments in the language classroom}
\abstract{While standardized assessments play an important role in understanding and measuring overall second language proficiency, instructors are often looking for additional ways to measure student proficiency in a way that better reflects the classroom practices and interactions of their students. This chapter looks at several alternative assessments to better understand students’ abilities in the target language as well as their overall proficiency. The first section of this chapter looks at the use of community-based language learning (CBLL) as a means to take students out of the classroom and provide them with opportunities to use the target language in meaningful context while serving within the community. The second section focuses on the use of Integrated Performance Assessments (IPAs) as an alternative to traditional assessments. IPAs provide students with the opportunity to include the three modes of communication: interpretive, interpersonal, and presentational in their assessments. IPAs allow students to better demonstrate their overall learning across all of the language modalities as well. The third section of this chapter analyzes the use of portfolios in the language classroom as an alternative to traditional assessments. Portfolios have been shown to offer students not only a way to gauge their progress and development but also a chance to reflect on their learning and plan for future language development. Finally, this chapter offers some concluding thoughts as well as the inclusion of an appendix with additional resources for developing these types of assessments and implementing them in the language classroom.
\keywords{assessment, portfolios, IPA, community-based language learning}
}

\begin{document}
\AffiliationsWithoutIndexing{}
\maketitle

\section{Introduction}
\largerpage[-1]
Foreign language assessment continues to be a key factor in understanding both student performance and proficiency. As language programs continue to develop and evolve, there has been a push to increase the accountability of such programs through improved assessment (\citealt{Bernhardt2006}; \citealt{Norris2006}). The increase in accountability is especially important in today’s current environment in the Humanities where language programs are shrinking or being eliminated \citep{Johnson2019}. This reduction in language programs is being done in spite of the fact that the need for competent language professionals has continued to grow and expand to meet an increasingly globalized world. The problem in the United States with many language programs at the college level is that despite valiant efforts, many students are graduating at the Intermediate High level or lower after finishing their degrees (\citealt{Rifkin2005}; \citealt{Magnan1986}; \citealt{Swender2003}; \citealt{Tschirner2016}). While Intermediate High level students do have some ability, the American Council on the Teaching of Foreign Languages defines this level of proficiency as being speakers who can deal with “routine tasks and social situations” and can handle “uncomplicated tasks and social situations requiring an exchange of basic information” (\citealt[7]{ACTFL2012}). This is not the level of proficiency needed to function and perform as competent language professionals in a wide variety of settings.

While standardized assessments of proficiency such as the Oral Proficiency Interview (OPI), The European Language Certificates (TELC), Japanese-Language Proficiency Test (JLPT), Diplomas de Español como Lengua Extranjera (DELE), etc. continue to be used and are valuable in comparing learners across a broad range of settings, more and more instructors are looking to alternative forms of assessment and evaluation of their students that better reflect the classroom practices and procedures as well as the preparation of their students to meet the current demands of language professionals. Instructors are also looking for ways to assess a more diverse student body who make up the fabric of many language classrooms and represent a wide range of cultures as well. In addition, there has been an emphasis over the last several decades on assessments that integrate culture into the language curriculum (\citealt{Bennett1986}; \citealt{Byram1997}; \citealt{Pedersen2010}; \citealt{Schulz2007}). \citet{ByrnesNorris2010} highlight the disconnect that exists between language and culture learning in foreign language programs at all levels and the lack of programs that develop translingual and transcultural students. \citet[120]{Sykes2017} discusses the importance of developing a transnational languaculture “in which language and culture transcend national boundaries, are uniquely tied to individuals (not only particular languages or cultures), and develop across a lifetime as learners move between a variety of contexts, locations, and languages”.

In order to help language students move in the direction of greater proficiency and intercultural competence, different types of assessments may be needed to push students into areas where they are better able to develop these skills. \citet{StigginsChappuis2006} state that the paradigm of assessment needs to change from one of the assessment OF learning to an assessment FOR learning. They declare, “Assessment for learning happens in the classroom and involves students in every aspect of their own assessment to build their confidence and maximize their achievement” (\citeyear[11]{StigginsChappuis2006}). \citet{BrownThompson2018} highlight three challenges faced in implementing changes in the overall assessment structure of many language programs referring specifically to Spanish programs:

\begin{quote}
The current status of assessment in many collegiate Spanish programs at the course and program levels is riddled with ironies:
(1) Many instructors are interested in student learning and are sure it is taking place, but are unclear how to validly demonstrate it;
(2) the primary mechanism accepted by key stakeholders (e.g., administrators, donors, and parents) to demonstrate effectiveness is through valid assessment, yet many instructors resist attempts to improve it or incorporate it; and
(3) persistent complaints from faculty about top-down mandates imposed by external parties such as accrediting agencies precede stiff resistance to take ownership of the process. (\citeyear[137--138]{BrownThompson2018})
\end{quote}

This chapter will briefly discuss three types of alternative assessments that can be used as tools of assessment for learning and that can be incorporated into a wide variety of language classrooms-community-based language learning (CBLL), Integrated Performance Assessments (IPAs), and portfolios. \citet{PierceOMalley1992} define alternative assessments as methods for determining student understanding and growth, informing, and leading to changes in teaching, criterion-referenced, authentic, integrating multiple language skills, and consisting of a variety of non-traditional assessments including teacher observation, performance assessments and self-assessments (\citeyear[4]{PierceOMalley1992}). \citet{TedickKlee1998} further describe how alternative assessments evaluate students:

\begin{quote}
Alternative assessments are not only designed and structured differently from traditional tests, but are also graded or scored differently. Student performance is evaluated on the basis of clearly defined performance indicators, criteria, or standards that emphasize students' strengths instead of highlighting their weaknesses. (\citeyear[3]{PierceOMalley1992})
\end{quote}

These assessments can help educators develop a clearer idea of the different abilities and overall learning of language students as well as address how diverse populations can benefit from thinking outside of the traditional assessment box.

\section{Community-based language learning}

Many educators are looking for ways to involve language learners in a broader community and use their language skills towards advancing the public good while creating informed citizens and community members. \citet{BringleHudson2004} pose the following questions regarding the responsibility of higher education in the formation of students.

\begin{itemize}
\item How can the challenge of educating future generations include socially responsive knowledge in a manner that is pedagogically sound?
\item How can undergraduate education prepare students for active participation in democratic processes in their communities?
\item How can students acquire the philanthropic habits that will enrich their lives and contribute to their communities? (\citeyear[3]{BringleHudson2004})
\end{itemize}

One of the ways to engage students and address these questions is through community-based learning (CBL) which falls under the broad umbrella of ex\-pe\-ri\-en\-tial-based learning. According to \citet{MooneyEdwards2001}, “Community-based learning refers to any pedagogical tool in which the community becomes a partner in the learning process” (\citeyear[182]{MooneyEdwards2001}). \citet{CliffordReisinger2019} further specify community-based learning: “Broadly speaking, community-based learning (CBL) serves as an umbrella term for activities that engage students within their communities and is often equated with service learning” (\citeyear[5]{CliffordReisinger2019}). CBL is focused on the concept of working with community partners in a collaborative relationship. \citet{Jacoby2015} found community engagement to be a high-impact educational practice that increases “the odds that students will invest time and effort; participate in active challenging learning experiences; experience diversity; interact with faculty and peers about substantive matters; receive more frequent feedback; and discover the relevance of their learning through real-world experiences” (\citeyear[11]{Jacoby2015}). While CBL has many similar features to other types of experiential learning such as internships, field work, volunteerism, or community service, it has certain distinct features that separates it from these other types of learning. \citet{Jacoby2015}, referring specifically to service learning, defines it as “a form of experiential education in which students engage in activities that address human and community needs, together with structured opportunities for reflection designed to achieve desired learning outcomes” (\citeyear[1--2]{Jacoby2015}).

While CBL is used across a variety of academic subjects, foreign language programs are increasingly working to establish community-based language learning (CBLL) experiences that focus on the acquisition of second languages. CBLL follows the same principles as CBL but concentrates on community-based learning that is designed for the language classroom.  Given the increase in programs employing CBLL, questions arise regarding not only how to successfully assess students but how CBLL relates to student learning outcomes (SLOs). Researchers have investigated using CBLL in the language classroom in order to help improve programs and students’ learning. \citet{Norris2006} writes that:

\begin{quote}
Assessments are only good insofar as their use does good, in terms of supporting educational efforts and outcomes. \ldots\,Where they do not obviously support the twin goals of helping educators deliver better programs and of helping students achieve valued learning outcomes, assessments should not be used. However, in order to realize these goals fully, assessments must be used. (\citeyear[582]{Norris2006})
\end{quote}

One of the key components to CBLL is meaningful reflection which can be challenging to assess in a way that is valid and reliable. \citet{Thompson2012} studied several advanced Spanish language classes and found that the inclusion of a CBLL component resulted in slight grade inflation since the grades were based almost entirely on simply participating in the different projects and not dependent on the quality of the CBLL. This resulted in some students raising their scores in the class by a full letter grade. He suggests that instructors need to measure “the quality of their service and reflection during the course of the class” (\citeyear[112]{Thompson2012}) and not simply grade based on the completion of the project or a certain number of hours. There are many ideas on ways to more empirically measure student gains during CBLL, but these would be contingent on the SLOs of a particular course and program. Thompson suggests having students bring the information back into the classroom and present what they have learned as a way to assess their learning more impartially. Educators could have students do presentations before and after their CBLL experiences and compare how their cultural and linguistic knowledge has changed using detailed rubrics. Depending on the focus of the class, these presentations could be done in the target language (TL) or in the first language (L1). A conversation course focusing on speaking could have a rubric designed to look more carefully at a student’s oral expression including elements such as pronunciation, vocabulary, grammar, etc. In a conversation course, the students could initially present on the organization where they would be doing their CBLL and some of the challenges that they expect to face as well as what skills they are bringing to the experience. The post-CBLL presentation would then result in the student sharing what has been learned from the experience and the instructor would again focus on the spoken aspects of the language.

In a literature or culture class, the decision to carry out the assessment in the TL would depend on the level and the overall objectives of the course. The rubric for such a presentation would likely focus much more on the content of the presentation and the ability of the students to organize their presentations in a compelling manner of interest to the instructor and fellow students. These presentations could be less focused on overall proficiency and performance and much more on the acquisition of intercultural competence or being able to make connections between the CBLL experience and the literature being studied. Writing assignments where students compare the situation of their community partners to their own lives could also benefit them not only from language acquisition and proficiency lenses but also help them become more aware of the situation of fellow community members. These writing assignments could be very similar to the oral presentations in the sense of whether to write them in the TL or L1. Instructors would need to again consider the course objectives, proficiency of the students, and goals for the CBLL assignment. For some courses, a thoughtful, reflective composition in the L1 might benefit the students more than having lower-level students struggle to present their reflections in the TL.\largerpage[-1]

\citet{MedinaGordon2014} investigated the role of using service learning during a language exchange between L1 English speakers and native Spanish speakers. The researchers developed a phonemic perception test that was used to measure these gains over the course of the semester. During weekly 60-minute sessions where the speakers would spend half the time practicing Spanish and the other half practicing English, adult college students were able to improve their phonemic perception. Given the use of a control group that did not use service learning, students who participated in service learning did show significant improvement over the course of the semester when compared to those who did not participate. Additionally, these researchers used a modified version of \citegen{GardnerMasgoret1997} Attitude/Motivation Test Battery and found that students who participated in service learning also had significant increases in motivation over the course of the semester when compared to those who did not participate.

While the study by \citet{MedinaGordon2014} did not include diverse learners, \citet{LowtherPereira2015} looked exclusively at heritage learners of Spanish who were participating in CBLL. She took a critical pedagogy approach and assessed her students’ overall development through detailed self-evaluations, reflections, questionnaires, and interviews. She found that through the CBLL, the heritage language learners developed a greater “awareness of sociolinguistic and sociopolitical issues affecting local Latino communities” and were better able to construct “positive identities” (\citeyear[159]{LowtherPereira2015}). \citet{Salgado-RoblesLamboy2019} also worked with heritage learners who were pre-service teachers assigned to different schools throughout the New York City region.

The students were evaluated based off six different assignments. First, they needed to complete 30--35 hours of service in their assigned schools. They received full points for completing the hours for this assignment. Second, they completed four “checkpoint” assignments during the semester.\largerpage[-2]

\begin{enumerate}[label={(\arabic*)},align=left]
\item A general description of the school and the community where the community service learning (CSL) project was being conducted, including ethnic and/or racial distribution in the school and the community; school offerings; languages taught; school rating; and personal, cultural, and community assets of the students in the selected class. (Due by the fourth week of the semester.)

\item A description of the need identified in the classroom, rationale for selecting this need, an action plan for the entire semester, and an explanation of how this intervention was expected to impact heritage speakers of Spanish. This had to be negotiated with and approved by the cooperating teacher (CT). (Due by the sixth week of the semester.)

\item A progress report that explained what the teacher candidate (TC) had done so far and a reflection on the CSL experience and its impact on student learning. This report had to address both positive and challenging (if any) aspects of this experience. (Due by the tenth week of the semester.)

\item A general assessment (reflection) of the TC’s own personal experience in this classroom, an evaluation of the project’s successes and challenges, and recommendations on how the identified need should be addressed in the future. (Due by the fourteenth week of the semester.) (\citeyear[1062--1063]{Salgado-RoblesLamboy2019})
\end{enumerate}

In the case of this class, the focus was more on the content of the experience and completing all of the components of each of the written assignments. Even though all the participants were heritage speakers of Spanish and working in Spanish-language classrooms with other Spanish-speaking students, these students were allowed to complete these activities in English since this was an education course. These same types of activities could also be developed for language courses focusing on developing the written proficiency of the students while completing these “checkpoint” assignments in the TL and having the instructor provide feedback on the language use and structure within the writing assignments.

The sixth and final assessment of the work by Salgado-Robles and Lamboy was a post-survey of their experience which was graded on the overall reflection and completion of the survey that can be found in the appendix of their article. This final survey was again written in English and mainly consisted of students selecting a number that best matched their feelings regarding the statements. The final part of the survey was composed of five open-ended questions to which the students were able to respond in English or Spanish. The researchers found through this service-learning experience that students developed a better understanding of what it means to be a teacher and what the profession entails. The participants were also better able to see the relevancy of the material from the class to their chosen profession. However, unlike the results from \citet{LowtherPereira2015}, the results did not show any impact on the participants’ view of their identities.

Both assessment by community partners and self-assessment of experiences in CBLL can serve as valuable sources for evaluating students. Regarding using community partner evaluations, \citet{BrownThompson2018} state:

\begin{quote}
Although such evaluations can be problematic, given the tendency of community partners to appreciate any help that is given, these partners can be provided with targeted, confidential online surveys where they can evaluate or even rank the students who worked with them. (\citeyear[90]{BrownThompson2018})
\end{quote}

Community partners are helpful in the evaluation of the students because they work with them and are able to help recognize their strengths and struggles as well as provide the instructor valuable with information on how to better prepare students for their CBLL experience. The authors  also recommend using peer evaluations as part of an overall picture of students’ performance during CBLL projects.

One way in which educators can employ a more empirically based self-re\-flec\-tion was outlined by \citet{AshClayton2009} through their DEAL model. While the DEAL model was designed for CBL, it can be applied to the language classroom through focusing the activities and assessments on improving the students’ abilities in the TL. The DEAL model consists of three different steps defined as Describe, Examine, and Articulate Learning. The first step is writing an objective and detailed description of the CBLL experience. The goal of this step is to help students see and describe their experience without the critical lens and is preparatory for more in-depth critical thinking. This could be carried out in the target language since using descriptive language can be adapted for different levels of proficiency. The instructor should develop a clear rubric for the evaluation of the students’ description and provide feedback.

The second step is to examine the experience beyond just summarizing what happened and trying to look at the relationship between civics and learning. This step is designed to help diverse learners understand issues of privilege and power, compare the individual and public good, and explore the dynamics of agency. Given the complexity of this examination, instructors may consider allowing students to write this reflection in the L1. Assessing this step would then need to focus on the students’ attention to detail, insights, quality of expression, etc. Instructors may consider having students record a presentation based off this step and share it on their learning management system for other students to view and comment.

Finally, the last step is the articulation of learning in which the learners develop goals for “future action that can then be taken forward into the next experience for improved practice and further refinement of learning” (\citeyear[42]{AshClayton2009}). These goals can be written down and then shared with the class with the students explaining their choice regarding the different goals. \citet{CliffordReisinger2019} state that this final step allows the learner to answer four important questions: “(1) What did I learn? (2) How did I learn it? (3) Why does it matter? and (4) What will I do in light of it?” (\citeyear[71]{CliffordReisinger2019}). CBLL provides the ideal environment to allow students to not only use their language skills but see how it directly impacts the community and specific individuals.

Summarizing the benefits of CBL, \citet{CliffordReisinger2019} declare that CBL “provides opportunities to expand interpersonal, intrapersonal, and cognitive domains in student development. Students learn more tolerance for ambiguity, dismantle stereotypes, build compassion, and establish reciprocal and authentic relationships” (\citeyear[28--29]{CliffordReisinger2019}). This is also inclusive of diverse communities who benefit from these interactions and reflections not only about their own language skills but also about the community and culture that surround them.

\section{Integrated Performance Assessments (IPAs)}

  According to \citet{Wiggins1998}, “the aim of assessment is primarily to educate and improve student performance, not merely to audit it” (\citeyear[7]{Wiggins1998}). Additionally, student assessment should be structured around authentic, real-life activities that are interactive and engaging for learners. One movement to try to achieve these goals has been through the greater use of Integrated Performance Assessments (IPAs). IPAs are defined as “ongoing, formative, and standards-based assessments that connect what is taught to what is learned and assessed and that provide the student with detailed and appropriate feedback” \citep{Adair-HauckTroyan2013}. \citet{DiazMaggioli2020} further describes IPAs as “a form of cluster assessment which capitalizes on the inherently intertwined nature of the three modes of communication: interpretive, interpersonal, and presentational” (\citeyear[24]{DiazMaggioli2020}).

The history of IPAs goes back to a project carried out by American Council on the Teaching of Foreign Languages (ACTFL) that used federal funding to design the IPA prototype in response to a high demand for standards-based assessments. The IPA prototype was designed to measure students’ progress towards reaching the ACTFL World-Readiness Standards. IPAs were created to assist instructors in connecting standards-based classroom instruction and assessment practices, so the two continuously coincided in the language classroom. The IPA prototype was to serve as a catalyst for curricular and pedagogical reform. ACTFL wanted to show educators how to properly connect assessment with practice so that they were not seen as separate identities in language learning.

Early research into performance assessment pre-date the development of IPAs and the integration of the ACTFL World-Readiness Standards. \citet{PierceOMalley1992} looked at the value of using performance assessments with language minority students. They describe how using a variety of performance assessments with diverse learners not only helped them to increase their participation but also improved the assessment of their learning. They conclude stating, “To be able to effectively monitor the progress of language minority students, assessment needs to be conducted on an ongoing basis with procedures that promise to yield the most useful information for classroom instruction” (\citeyear[27]{PierceOMalley1992}).

\citet{DiazMaggioli2020} describes current research into using IPAs and how to help students increase their performance. He found that students often struggle with interpretive and interactive tasks due to their lack of exposure to authentic listening sources and opportunities to interact with native speakers. Frequently students are only receiving input from their instructors and from modified audio sources. In addition, most of their conversation are with fellow second language learners who often struggle with the same issues that they have and are not able to help them make the necessary corrections to their speech.

In addition to IPAs helping students learn, \citet{Adair-HauckEtAl2006} conducted a study of over 1000 students as well as 30 foreign language instructors to determine the impact of IPAs on the instructor’s perception of learning. The participating instructors reported that the IPAs:

\begin{quote}
served as a catalyst to make them more aware of the need to integrate the three modes of communication into their lessons on a regular basis, design standards-based interpretive tasks using authentic documents, integrate more interpersonal speaking tasks, use more open-ended speaking tasks, and use more standards-based rubrics to help the students improve their language performance. (\citeyear[373]{Adair-HauckEtAl2006})
\end{quote}

Thus, the implementation of the IPAs helped to make the instructors not only more aware of what they needed to be doing in the classroom with their students but also helped to focus them on a more standards-based approach to language learning. The IPAs are able to move students and instructors from viewing language as the acquisition of a single skill to an interconnected approach of integrating the different language modalities together to acquire a language.

\citet{Troyan2016} states that in selecting the appropriate listening segments and reading passages that educators should consider two important factors “(1) learn\-er-based factors (e.g., linguistic level and age) and (2) text-based factors (e.g., context and the task related to the text)” (\cite[171]{Troyan2016}). Considering these factors can help guide instructors to be more decisive in the materials that they use in their classes and improve their students’ experiences with them. \citet{Adair-HauckTroyan2013} mention the following resources as examples of where to find authentic sources appropriate for specific learners:

\begin{itemize}
\item Interviews or surveys from youth-oriented TV programming;
\item \begin{sloppypar}Straightforward conversations taped from a youth-oriented music pro\-grams on TV or radio;\end{sloppypar}
\item Product commercials in the target language from TV or radio;
\item Public service announcements on radio or TV such as anti-smoking or anti-drug campaigns;
\item Authentic songs by artists of the target culture based on familiar contexts or theme being studied;
\item Animated cartoons;
\item Segments from soap operas or other television programming;
\item Interviews from talk shows from the target culture. (\citeyear[34]{Troyan2016})
\end{itemize}

IPAs can also include an element of CBLL where students can interact with native speakers and reflect on these interactions. Since IPAs focus on interpersonal, interpretive, and presentational speaking, using CBLL highlights the interactive, sociocultural nature of language learning and moves it beyond just listening and understanding (input).

The benefit of this type of assessment for learners is the exposure not only to authentic sources and speakers, but an opportunity to engage with diverse communities and develop intercultural awareness and sensitivity. One can imagine a classroom where the students are participating in CBLL and thus receiving authentic input and engaging with native speakers in the second language. They are then coming back into the classroom where they are presenting and reflecting on their interactions as well as considering some of the struggles, which they have had both with the language and with any cultural misunderstandings through IPAs. These students can then work with their instructor to practice the areas where they need to improve and develop a plan based off the three modes of communication that they are using. This type of constant and constructive feedback and performative assessment would assist the students in understanding their own language development and growth.

\citet{Adair-HauckTroyan2013} summarize their research on IPAs declaring, “The IPA provides useful information to both the teacher and the learners regarding the kinds of authentic tasks the learners can perform across the three modes of communication and what the learners need to do to improve their language performance” (\citeyear[37]{Adair-HauckTroyan2013}). In order to assess students’ growth and development with the IPAs, ACTFL’s performance descriptors would be excellent criteria that could be used with different levels of proficiency depending again on the class level. These performance descriptors could be used with the different phases of the IPA to help students with more formative assessments. ACTFL also has the can-do statements that could serve as a baseline for measuring what students are able to do and could be applied to some parts of the IPA. Finally, the Common European Framework of Reference for Languages (CEFR) also has common reference levels that would be good criteria for looking at student growth especially in regards to language proficiency.

\section{Portfolios}

  Portfolios have existed and been widely utilized in many professions such as art, architecture, photography, journalism, etc. \citep{Lam2017}. The use of portfolios in the language classroom also has a rich tradition that has continued to evolve as technology changes the ways in which they are developed and presented \citep{Fox2016,Lam2017,McMillan2018}. \citet{McMillan2018} defines portfolios as a “purposeful, systematic process of collecting and evaluating student formative and/or summative assessments to document progress toward the attainment of learning targets or show evidence that learning targets have been achieved” (\citeyear[303]{McMillan2018}). One of the keys of portfolio assessments is the ability to demonstrate progress (formative assessment) of students over a period of time even though they can be used for summative assessment as well. \citet{Hamp-LyonsCondon2000} declare that portfolios involve three phases: collection, selection, and reflection.

There are several benefits to using portfolios as a tool for assessment. They help create a match between classroom activities and assessment. Students will be able to better understand what is going on in the classroom and then be assessed in a way the reflects their learning. They also capture a rich array of what students know and can do without focusing too much on what students cannot do as in many traditional assessments. Along these same lines, portfolios chronicle students’ language development over time and show their progress. This allows students to highlight where they started from and where they have reached \citep{GeneseeUpshur1996}. This allows for more differentiated assessment since summative assessments often do not recognize growth especially in struggling students.

Portfolios also allow students to evaluate their own work, effort, strategies, goals, and progress as these assessments require self-assessment and reflection. Students are able to explain their growth and take responsibility for their own learning. They are also able to better understand how grades are represented as they compile and consider their own portfolios. Since portfolios are often formative, they allow students to establish ongoing goals and review their progress towards the goals they have established \citep{TedickKlee1998}.

In portfolios, students are able to demonstrate their overall proficiency both in regards to language and culture and portfolios can even empower students to become their own advocates for their learning as well as for their assessment \citep{AlamAkar2019}. Portfolios allow students to explain their learning in a way that is collaborative in nature with their instructor leading to greater language acquisition. Finally, \citep{TedickKlee1998} explain that portfolios are not limited to one language modality but represent “a student’s range of performance in reading, writing, speaking, and listening as well as cultural understanding” (\citeyear[20]{TedickKlee1998}). The assessment of portfolios can involve a variety of individuals including peer assessment, self-assessment, collaborative assessment, and instructor assessment. \citet{TedickKlee1998} observe:

\begin{quote}
Determining how to go about assessing portfolios in a systematic way is a process that involves reflection, much discussion and negotiation with students and colleagues, and risk-taking. The more the collaboration, the better the process, and, most certainly, the outcome. (\citeyear[22]{TedickKlee1998})
\end{quote}

\citet{Lam2017} states that the rationale for implementing portfolio assessment is that assessment “should be personalized, longitudinal and contextualized, taking place in learners’ familiar classroom environments rather than being dehumanized and standardized, administered in the examination hall” (\citeyear[85]{Lam2017}). This personalization of learning is valuable to language learners and makes them feel part of the process of language acquisition. \citet{TedickKlee1998} declare, “The evaluative process should include ongoing (formative) assessments of students’ work as well as overall (summative) assessments” (\citeyear[21]{TedickKlee1998}).

\citet{McMillan2018} describes four types of portfolios that instructors can use to assess their students. He classifies three of the types as documentation portfolios (celebration/showcase, competence or standards-based, and project) with the other category being growth portfolios. All of these portfolios can be done in the TL and often are since they are related to the work in the course. The reflections can be completed in the L1 or TL depending on the level of the students and the goals of the course.

Celebration/showcase portfolios are compiled to show a student’s work that illustrates achievement and highlight some exceptional part of learning. In these cases, the student often selects their best work or what they are most proud of to share with the instructor and/or class. Since each student picks what information they want to highlight, each individual portfolio is unique and personalized to the individual. While this allows for a great deal of creativity and individuality, it also complicates the scoring of each portfolio and can make reliable scoring a challenge especially across a large classroom.

The second type of documentation portfolio that can be used to assess language students is a competence or standards-based portfolio. \citet{McMillan2018} defines this type of portfolio as being designed “to provide evidence that a targeted level of proficiency has been achieved. For this kind of portfolio, the criteria for determination of mastery or competence need to be clearly defined” (\citeyear[304]{McMillan2018}). The competence or standards-based portfolio is one that is designed to collect a wide range of evidence regarding the proficiency level of the students for a specific class or program. Evidence can be collected based off all language modalities and representative of the overall competence of a language learner.

The third type of documentation portfolio is the project portfolio. \citeauthor{McMillan2018} states that the main objective of these types of portfolios is to provide a “single example or illustration of the competence of the student” (\citeyear[304]{McMillan2018}). Students compile these portfolios with a very specific task in mind and work towards assembling these with the mindset of highlighting some specific aspect of their learning. The final type of portfolio mentioned by McMillan is the growth portfolio. The growth portfolio is a formative assessment that can be used to assess the changes in the proficiency level of students over time. These types of portfolios are beneficial in documenting changes in students’ language skills and also provide examples to allow students to see their own growth in the skill sets that they possess.

\citet{Wewer2020} studied how language portfolios were being used in the European context where the implementation of the European Language Portfolio (ELP) was developed in 2001 in cooperation with the Common European Framework of Reference (CEFR) for language learning. According the ELP website  (\url{https://www.coe.int/en/web/portfolio}), the European Language Portfolio (ELP) was developed by the Language Policy Programme of the Council of Europe \citep{CouncilofEurope2001} to support the development of learner autonomy, pluri\-lin\-gualism and intercultural awareness and competence; to allow users to record their language learning achievements and their experience of learning and using languages.

In spite of the implementation of the ELP, \citet{Wewer2020} found that out of all of the different types of assessments both traditional and alternative that “the least used assessment method by teachers was the language portfolio” (\citeyear[150]{Wewer2020}) even though it has been around since 2001. She also found that other alternative assessments such as simulations, peer assessments, and graded presentations were also among the least common assessments with most instructors opting for traditional assessments or standardized assessments. \citet{Wewer2020} further states that the lack of use of alternative assessments “calls the serious question of whether or not teacher-based, formative assessment is genuinely used for the purpose of enhancing learning” (\citeyear[150]{Wewer2020}). She discovered that instructors’ intentions for language assessment differed dramatically from their actual assessment practices. She declares:

\begin{quote}
The cornerstone of any approach to assessment promoting learning in CLIL is to make the learners aware of the dual learning objectives (content and language), their own learning processes, what is already learnt, and how they themselves can further promote and advance the attainment. Such an action necessitates communication and feedback. One means to this end could be the least used assessment method reported by teachers, the language portfolio. (\citeyear[160]{Wewer2020})
\end{quote}

In spite of its lack of generalized use in the European context, \citet{Wewer2020} notes that in classrooms using portfolio assessments both parents and students found them to be useful and enjoyable as well as being good representations of students' language skills. She also comments that those classes that employed language portfolios as part of their assessments had slight increases in their proficiency in the TL.

Regarding the assessment of portfolios, \citet{TedickKlee1998} recommend that instructors not only have their students compile the portfolio but also should ask students to reflect on question related to their overall understanding of their learning process and struggles. Students may be asked to reflect on their acquisition of language and culture as well as how they have contributed to their learning. The assessment of the portfolio would be based on a rubric to analyze both the quality of the portfolio and the reflection. \citet{Kunschak2020} suggests combining portfolio assessments with other testing measures. She recommends measures that could include:
\begin{quote}
standardized test scores but also evidence of achievement of learning outcomes such as papers or videos or other authentic samples of performance tied to a specific rubric of learning objectives (e.g., a term paper on a policy issue or a group presentation on a mini-research project). Evidence of progress such as multiple drafts, peer review sheets, and reflective comments or diagnostic, mid-term, and final in-class timed writings could also be included. (\citeyear[99]{Kunschak2020})
\end{quote}

She goes on to say that “evaluations of innovative programs need to be cyclical like action research, moving from planning to implementation, assessment to reflection and on to the next round” (\citeyear[99]{Kunschak2020}).

\citet{DelettKevorkian2001} provide several steps necessary for successful portfolio assessment. They state that the first step is to plan the assessment purpose. This coincides with the different types of portfolios mentioned earlier in this chapter where an instructor needs to determine the reason for choosing a portfolio assessment. The second step is to define the portfolio outcomes as these are important to help the learners and instructor focus on the skills and knowledge that they hope students will acquire. The third step is to match the classroom activities with the established outcomes. Since the portfolio assessments often consist of assignments from the class, it is important to make sure that the classroom activities generate the necessary materials for the portfolios and that they are articulated to maximize learning. The fourth step is to establish the organization of the portfolio. Having confusing instructions or not being clear regarding the content of the portfolio can make this a negative experience for the students as they will not be sure what to include in their portfolios. Fifth, the instructor needs to clearly establish grading criteria so that students know what is expected and what represents excellence in their portfolios. Establishing clear rubrics for every aspect of the portfolio assessment will make the overall grading both transparent and fair. Sixth, the instructor needs to make sure to monitor students’ progress throughout the whole process. If an instructor waits until the end of the unit or course to finally see the portfolios, it is likely that some students will have misunderstood and at that point, it will be too late to make meaningful changes. Finally, the instructor needs to monitor and reflect upon the whole portfolio process not only to make changes during the semester but also from semester-to-semester and year-to-year. As certain assignments work and others do not then the instructor needs to be cognizant of the needed changes and make them. All of these steps can lead to making portfolio assessment a valuable tool to understand students’ growth and learning in the language classroom. Portfolio assessment can also help the instructor to see how daily activities in class need to reflect the overall language learning objectives.

\section{Conclusion}

\begin{sloppypar}
  Briefly outlined in this chapter are three alternative assessments which can greatly serve diverse learners with more contextualized settings for learning and assessment (see Appendix for further resources). Using CBLL, IPAs, and portfolios can move assessment from simply recalling and repeating information gleaned from classes and readings to the real application of language skills and abilities with authentic communities and through authentic resources. As \citet{Kunschak2020} states:
\end{sloppypar}

\begin{quote}
By integrating content and language in assessment, students can be subtly guided towards a more holistic approach to learning or deep learning with a view to applying their skills rather than studying vocabulary for a test or memorising concepts by heart. (\citeyear[98]{Kunschak2020})
\end{quote}

While the scope of this chapter only allows for a sampling of the many ways these assessments can be used both in and out of the classroom to promote student learning and development, these should help instructors by providing them with some ideas on where to begin to implement changes in the way students are assessed. Additionally, instructors can also consider the many ways in which all three of these alternative assessments can be complimentary to each other and could be used together as both formative and summative assessments of students’ language and culture development. Keeping the students learning outcomes in mind, instructors can revisit their current forms of assessment and determine where they may be able to make changes to better help their students become more competent language learners.



% \begin{paperappendix}

\section*{Appendix: Additional resources for alternative assessment development}

This following appendix contains additional resources with more information, rubrics, additional examples, and other help needed for developing these tools to be used in the classroom.

\subsection*{Community based language learning}

\begin{itemize}
\item Clifford, J., and Reisinger, D. (2019).~\textit{Community-based language learning: A framework for educators}. Washington, DC: Georgetown University Press. DOI: \href{https://doi.org/10.2307/j.ctv7cjw41}{10.2307/j.ctv7cjw41}.\textbf{ -- }This book provides several models of CBLL that could be employed in a variety of language classes. It contains numerous examples as well of different types of CBLL.
\item Salgado-Robles, F., and Lamboy, E. M. (2019). The learning and teaching of Spanish as a heritage language through community service learning in New York City. \textit{Revista Signos. Estudios de Lingüística}, \textit{52}(101), 1055--1075.\textbf{ -- }This article contains appendices with additional materials for assessing students learning through CBLL.
\item Thompson, G. L. (2012). \textit{Intersection of service and learning: Research and practice in the second language classroom.} Charlotte, NC: Information Age Publishing.\textbf{ -- }This book focuses specifically on service learning and provides examples and materials for developing programs and integrating this into the language classroom.
\item \url{https://uca.edu/servicelearning/faculty/assessment-3/} \textbf{-- }This website from the University of Central Arkansas provides links to different rubrics and materials that can help in assessing CBLL.
\item \begin{sloppypar}Baker, L. (2019). Community-based service-learning in language education: A review of the literature. \textit{International Journal of Research on Service-Learn\-ing and Community Engagement}, \textit{7}(1), Article 2. \url{https://ijrslce.scholasticahq.com/article/11480.pdf} \textbf{--} This comprehensive review of previous studies from 1997--2017 can help instructors understand best practices in teaching and assessing community-based learning.\end{sloppypar}
\item Bloom, M, and Gascoigne, C. (Eds.). (2018). \textit{Creating experiential learning opportunities for language learners: Acting locally while thinking globally}. Multilingual Matters. \textbf{--} This book focuses on domestic experiential learning experiences for language learners providing examples of many different types of programs that could be implemented.
\item Tocaimaza-Hatch, C. C., and Walls, L. C. (2016). Service learning as a means for vocabulary learning in L2 and heritage language learners of Spanish. \textit{Hispania, 99}(4), 650--665. \textbf{--} This article looks at how to assess vocabulary through service learning with diverse learners.
\end{itemize}

\subsection*{Integrated Performance Assessments}

\begin{itemize}
\item \url{https://carla.umn.edu/assessment/vac/CreateUnit/p_2.html} \textbf{--} This website has step-by-step instructions on how to implement and design IPAs as well as examples of IPAs from different languages.
\end{itemize}
\begin{itemize}
\item Adair-Hauck, B., Glisan, E. W., Koda, K., Sandrock, S. P., and Swender, E. (2006). The integrated performance assessment (IPA): Connecting assessment to instruction and learning. \textit{Foreign Lan\-guage Annals, 39,} 359--382.\\\textbf{-- }This article has several appendices with materials useful for developing IPAs.
\end{itemize}
\begin{itemize}
\item Adair-Hauck, B., Glisan, E. W., and Troyan, F. J. (2013). \textit{Implementing integrated performance assessment}. Alexandria, VA: ACTFL. \textbf{-- }This book was written to provide step-by-step details on how to design, implement, and assess IPAs.
\end{itemize}
\begin{itemize}
\item \begin{sloppypar}Adair-Hauck, B., and Troyan, F. J. (2013). A descriptive and co-constructive approach to integrated performance assessment feedback. \textit{Foreign Lan\-guage Annals}, \textit{46}(1), 23--44. \textbf{-- }This article has several appendices with materials useful for developing IPAs.\end{sloppypar}
\end{itemize}

\subsection*{Portfolio assessments}

\begin{itemize}
\item \url{https://www.coe.int/en/web/portfolio} \textbf{--} This is a link to the European Language Portfolio site which has many resources on how to develop and assess language portfolios.
\item \begin{sloppypar}\url{https://www.pinterest.com/cchwedor/fsl-cefr-european-language-portfolio/} \textbf{--} This Pinterest board has many samples from the European Language Portfolio that teachers can use as models.\end{sloppypar}
\item Delett, J. S., Barnhardt, S., and Kevorkian, J. A. (2001). A framework for portfolio assessment in the foreign language classroom.~\textit{Foreign Language Annals},~\textit{34}(6), 559--568. \textbf{--} This article has several appendices with materials useful for developing portfolio assessments.
\end{itemize}
% \end{paperappendix}
\sloppy
\printbibliography[heading=subbibliography,notkeyword=this]
\end{document}
