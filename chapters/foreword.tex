\documentclass[output=paper]{langscibook}
\ChapterDOI{10.5281/zenodo.7304934}

\author{Marleen {Haboud Bumachar}\affiliation{Pontificia Universidad Católica del Ecuador; Ruhr-Universität Bochum}}
\title{Foreword}
\abstract{}
\begin{document}
\maketitle

\noindent\textit{Language assessment in multilingual settings: Innovative practices across formal and informal environments} edited by Eva Rodríguez-González and Rosita L. Rivera is a very important contribution, not only for academics interested in a quality educational system that erases social inequities, but above all, for students, second language learners, heritage speakers, or multilingual and multicultural speakers who struggle to adjust to a homogeneous monocultural system, divorced from their reality\slash social context characterized by diversity. In this sense, this book strives to join forces to build, from school and beyond, a more conscious and fair society that understands and accepts difference, not as a barrier, but as a source of wealth as an individual and collective asset.

Each chapter draws our attention to the urge to create awareness towards the student body heterogeneity and, therefore, the need to rethink and re-create the curriculum contents, the teaching-learning methodologies, as well as the urgency of leaving behind assessment practices that look at\slash conceive students as homogeneous entities, to instead move on towards developing strategies that take into account their particularities, abilities, needs and expectations. 

This book focuses on heritage speakers who look forward to rediscovering the language of their parents, the wisdom of their ancestors, and to proudly reinforce their identity. It also reminds us of the important role we have when assessing students from different linguistic and sociocultural backgrounds, or when we act as cultural interpreters of speakers of less prestigious languages who often confront harsh living conditions.

The varied topics discussed in each chapter bring us closer to multiple territories, languages, cultures and identities, with a broad perspective and alternative paths that move us to delve into diversity with creativity and respect, and with the conviction that multilingualism must be a tool to glimpse at social justice. Indeed, each one of the authors opens up a range of strategies and possibilities for us to contribute towards building a more equitable society, based on respecting the difference. 

Anyone who reads this book will deeply feel the urgency to deeply scrutinize their own autobiography in order to have a better understanding of the self and the other, and the many others, who have been made invisible in many of the classrooms. Each reader will be inspired to see diversity through new lenses, and to develop creative strategies to dialogue with the richness and the challenges fetched by diversity.

Anyone who is genuinely interested in walking hand in hand with their students, in understanding the profound meanings of every one of their words and gestures, in moving away from automated evaluation or mechanical translation towards intercultural dialogues that pursue social justice, will not be able to stop reading this book again and again. Near the end, the reader will not only be inspired, but truly committed to contributing in the search for \textit{equity in diversity}.

\end{document}
