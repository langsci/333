\documentclass[output=paper]{langscibook}
\ChapterDOI{10.5281/zenodo.6762284}

\author{Rosita L. Rivera\affiliation{University of Puerto Rico-Mayagüez} and Eva Rodríguez-González\affiliation{University of New Mexico}}
\title{Pedagogical implications of assessment in multilingual contexts}
\abstract{This chapter discusses the case of multilingualism and context-based approaches to assessment as an eclectic approach that requires a robust knowledge and understanding of the linguistic diversity of language learners. It also calls for a more inclusive approach in the design and implementation of assessment in higher education. We also connect the findings from the contributions to this volume and their research to an ecological approach to assessment as it relates to linguistic sensitivity in multilingual contexts. The chapter ends by suggesting future directions for research based on alternative assessment and connecting those to current issues in language learning and assessment.}

\IfFileExists{../localcommands.tex}{%hack to check whether this is being compiled as part of a collection or standalone
  \addbibresource{../localbibliography.bib}
  % add all extra packages you need to load to this file

\usepackage{tabularx,multicol}
\usepackage{url}
\usepackage{soul}
\usepackage{longtable, xltabular}
\urlstyle{same}

\usepackage{listings}
\lstset{basicstyle=\ttfamily,tabsize=2,breaklines=true}

\usepackage{langsci-basic}
\usepackage{langsci-optional}
\usepackage{langsci-lgr}

\usepackage{todonotes}

\usepackage{makecell}

\usepackage{enumitem}
\usepackage{multirow}
\usepackage{langsci-branding}
\usepackage{langsci-gb4e}

  \newcommand*{\orcid}{}

\newcommand{\togglepaper}[1][0]{
%   \bibliography{../localbibliography}
    \papernote{\scriptsize\normalfont
    \theauthor.
    \titleTemp.
    To appear in:
    Change Volume Editor \& in localcommands.tex
    Change volume title in localcommands.tex
    Berlin: Language Science Press. [preliminary page numbering]
  }
  \pagenumbering{roman}
  \setcounter{chapter}{#1}
  \addtocounter{chapter}{-1}
}

% \newcommand{\keywords}[1]{\par\textbf{Keywords: #1}}

\renewcommand{\lsChapterFooterSize}{\footnotesize}

  %% hyphenation points for line breaks
%% Normally, automatic hyphenation in LaTeX is very good
%% If a word is mis-hyphenated, add it to this file
%%
%% add information to TeX file before \begin{document} with:
%% %% hyphenation points for line breaks
%% Normally, automatic hyphenation in LaTeX is very good
%% If a word is mis-hyphenated, add it to this file
%%
%% add information to TeX file before \begin{document} with:
%% %% hyphenation points for line breaks
%% Normally, automatic hyphenation in LaTeX is very good
%% If a word is mis-hyphenated, add it to this file
%%
%% add information to TeX file before \begin{document} with:
%% \include{localhyphenation}
\hyphenation{
anaph-o-ra
Dor-drecht
Ku-ka-ma
pre-dom-i-nant-ly
prog-ress
teach-er
Ri-ve-ra
}

\hyphenation{
anaph-o-ra
Dor-drecht
Ku-ka-ma
pre-dom-i-nant-ly
prog-ress
teach-er
Ri-ve-ra
}

\hyphenation{
anaph-o-ra
Dor-drecht
Ku-ka-ma
pre-dom-i-nant-ly
prog-ress
teach-er
Ri-ve-ra
}

  %\togglepaper[]
}{}

\epigram{Assessment works when we learn to look at it as a process for improving the quality of our teaching. It works when we dialogue with colleagues, both within our discipline and across campus, and create new ideas to help students learn. Assessment works when we try something new and don’t get disheartened when it doesn’t work; instead, we reevaluate and try something else. Assessment works when something new proves effective and we gain information that moves our curriculum forward. Assessment can work if we quit making excuses as to why it’s so difficult and messy and instead look to the information to reinforce what works and discard what doesn’t.  Assessment works when we embrace the challenge of always getting better}
\epigramsource{Vickie Kelly (2017, Washburn University)}

\begin{document}
\maketitle



\section{Linguistic sensitivity and pedagogical training in language assessment}

As researchers continue to develop multiple ways to document learning growth and development, the different needs for teacher training continue to grow. Language program coordinators designing curricula for language courses may face many challenges if they choose to focus primarily on proficiency skills and standardized language testing. For instance, in classroom-based language scenarios, interpersonal and intercultural competence may be left behind or not prioritized in terms of learning outcomes and, as a result, the social aspect of language development might not be monitored or even paid that much attention for assessment purposes. In this regard, \citet{MenkeMalovhr2021} state that both the limitation and the challenge “is not necessarily in identifying the problem, but in allocating institutional resources (both human and financial) to revising and designing curricular sequences that systematically develop desired learning outcomes” (\citeyear[500]{MenkeMalovrh2021}). In this regard, \citet{PhakitiIsaacs2021} highlight the importance of assessment literacy and make a call to the scholarly community to empower teachers “to deal and communicate with external mandates such as government or state agencies who often impose external assessments on students and educational systems. For teachers, understanding assessment quality is more important than ever” (\citeyear[19]{PhakitiIsaacs2021}).  The authors recommend excellent resources for teachers (\citeyear[Appendix A]{PhakitiIsaacs2021}) and call them to be critical consumers of learning materials and assessment instruments \citep{BrownTrace2017}, including through professional development activities \citep{HardingKremmel2017}.

The integration and emphasis on linguistically responsive instruction that is both inclusive and offers a variety of opportunities for activities and assessment tools should be explicit and prioritized in language learning contexts. However, as \citet{HuangLaskowski2014} point out for English second language teaching, translating such a view of language education into classroom practice requires the instructor to be linguistically sensitive to both the content and tasks that learners face during their own learning path. While studies exist to show how effective language instructors integrate language and content and prepare future language speakers for the job market, research attention is still emerging on how instructors are trained to be equipped with the needed knowledge and skills on assessment techniques. Thus, it is important to consider how instructor-training may benefit from integrating awareness of multilingual learners' realities. The notion of an “ideal learner” or “test-taker” could be challenged through instructor-training that presents different ways of assessing language learning to encourage collaboration and linguistic mediation among learners. When referring to training where teachers also learn about language anxiety, the role of emotions in the language classroom, power dynamics, and agency in and outside of the classroom is key to assessment of languages. In this regard, the chapters by \textcitetv{chapters/2} and \textcitetv{chapters/3} in the present volume offer suggestions on the kind of activities and assessments that are sensitive to the multiple learner profiles within a given classroom setting. Thompson’s IPAs and language portfolios are examples of multidimension assessment tools that measure language performance and consider learner reflection as an ongoing process that documents language growth. Similary, the chapters by \textcitetv{chapters/3}  and \textcitetv{chapters/5} in this volume identify a specific learner survey (Can-Do statements survey) that allows both the learner and instructor to monitor learner self-efficacy and language development.

Because assessment is process-oriented, it has significant potential for exploring language and culture from an interdisciplinary and midway perspective. By including multiple modalities of assessment such as those done via self-as\-sess\-ment and questionnaires where learners and community members share perceptions of their own language use, experience and learning such as the ones used in this volume by Silva in her article on Brazilian Heritage Portuguese and Vallejos et al. when assessing Kukama and Kichwa, we acknowledge challenges in learning and embrace innovative instructional practices in response to cultural and linguistic diversity. Additionally, by involving peers and community members in assessment practices, we will be allowing spaces for creative interpretation that include individual and collective voices that engage with each other when monitoring language development and personal growth. Through critical, self-reflexive practices embedded in our research about language learning, teaching and assessment, we can work against racial, cultural, linguistic, and socioeconomic inequalities by creating humane classrooms and/or communities of practice where learners and instructors learn together to use language and literacy in critical and empowering ways.

When preparing and training future language instructors or accreditors that will be in charge of documenting and monitoring language assessment, educators should serve as advocates and models of social justice and equity. Social justice-oriented instructors and trainers play a significant role in seeking alternative ways to address various forms of official knowledge with the learner populations they serve, especially forms of official knowledge that marginalize certain groups while privileging others. For instance, language assessments should be shaped according to multiple heritage language profiles and second language learners. In this volume, \textcitetv{chapters/3} describes assessment tools and provides instructions for implementation in classroom settings for both heritage language speakers and second language learners. Additionally, the chapter by \textcitetv{chapters/4} on pragmatic development offers suggestions for assessing pragmatic language growth in study abroad language settings. By means of reviewing the existing literature on the topic, Hernández carefully examines how pragmatic knowledge can be measured and included as another key area in language learning and curriculum design. Instructors should also envision classrooms as safe spaces for struggle and transformative action and social change. For instance, community-based language learning and study abroad experiences (see \citetv{chapters/4}, \citetv{chapters/6} and \citetv{chapters/5}) offer a plethora of opportunities for creative and reflective assessment practices that allow learners to shape their own transformations and interconnect themselves with others in a more naturalistic way than what a traditional classroom and curriculum might offer.

In a 2005 position statement from the National Council of Teachers of English \citep{NCTEJuly312005} entitled “Supporting Linguistically and Culturally Diverse Learners in English Education”, the following beliefs and recommendations were suggested as a call for action. All of them may serve well as a point of reference for consideration for instructors, program coordinators and scholars working on alternative language assessments that are sensitive to the multiple realities of learners: (1) respect for all learners; (2) funds of knowledge; (3) inquiring into practice; (4) variety of educational experience; (5) modeling practice; (6) critical users of language; (7) crossing cultural boundaries; and (8) teaching as a political act. These recommendations from the \citealt{NCTEJuly312005} report, although more aligned for teaching per se, have implications for language assessment. For example, instead of creating language activities and assessments where accuracy in sentence formation and vocabulary use may be biased towards a variety or dialect of the target language as determined by power and prestige, instructors acknowledge and give credit for different ways of addressing people, different culture-driven practices to denote proximity and kindness as a way to respect all learners and speakers. How do instructors and teacher educators successfully integrate the funds of knowledge learners bring into their pedagogic and assessment stance? Reflection journals and presentations of community-based projects such as the ones identified by Thompson in the present volume open a myriad of opportunities for intercultural competence regardless of whether the learner has shared that reflection on the target language or not. The learner becomes his/her own agent and funds of knowledge for assessment. The assessment piece in learner’s reflections would not need to have a rubric or a scale of points for a grade, but rather credit should be confirmed via a checklist of the items requested by the instructor to be included in the reflection (subjective scoring based on evidence provided in the reflection as checked by both learners and instructor). For instance, the Can-Do statements survey used by Rodríguez-González et al.'s and Dickinson \& Martínez in this volume serves as an example of an assessment tool that uses reflection and is learner-driven.

\begin{sloppypar}
Educators may benefit from learning more about sociolinguistics both in teacher preparation programs and in ongoing professional development. Developing this kind of knowledge may help to avoid language marginalization \citep{DelpitKilgourDowdy2003}. By training future instructors in alternative assessments, they will have a positive impact on the attitudes towards those assessment (vs. traditional ones) and their own attitudes and beliefs about the nature of language learning may organically evolve as well when applying multiple ways and tools of assessment (see \citealt{Shahbari2018} for findings related to prospective Mathematics teachers on alternative assessments). Examples of professional development related to alternative assessments could include but not limited to workshops on the inclusion of portfolios as “bodies of evidence” to document learner’s individual paths and growth in multiple communities of practice that differ substantially from the traditional classroom (see \citetv{chapters/2} for Portfolio recommendations and \citegen{Green2014} PRICE principles for promoting effective classroom assessment- Planning, Reflection, Improvement, Cooperation, and Evidence).
\end{sloppypar}

\section{An ecological approach to assessment}

As we finished this volume, we were in the middle of a pandemic that challenged language instruction and assessment in all contexts. Language practitioners are now faced with new approaches to dealing with a crisis. The pandemic is one example that challenged everyone, but there are also multiple contexts dealing with other issues such as natural disasters. Learners and educators may not have access to the same sources, which creates a disparity between formal and informal assessment \citep{MenkeMalovhr2021}. These disparities are largely due to factors ranging from design of formal and informal assessment to measuring proficiency. Yet, learners' futures depend on the design and implementation of these instruments. \citet{Ortega2017} and \citet{MazakCarroll2016} make a case for challenging the ontological view of monolingual ideologies in language research and practice. New lines of inquiry are also undertaking an ecological orientation in language learning and assessment in SLA \citep{Larsen-Freeman2017}. This epistemological view of a dynamic and contextual approach to assessment could also be applied to foreign language instruction in multilingual contexts. This perspective acknowledges that language learning and assessment do not take place in isolation from the temporal space in which they occur, but they are dynamic in nature and change with the environment \citep{Larsen-Freeman2017}. This perspective not only considers the traditional and individual factors such as age, motivation, aptitude and attitude, but also considers the underlying issues related to the learners as individuals in constant interaction with their historical and sociocultural, and sociopolitical contexts. Examples of these are issues related to learners' anxiety, emotions, values and beliefs about the language. In this volume, programs such IMPACT (see \textcitetv{chapters/5}) and the FORMABIAP \parencitetv{chapters/7} serve as a couple of examples of projects that include sociolinguistic profiles of learners for pedagogical and career-decision purposes. Such programs address the nature of the environment and the learners’ willingness to communicate. Assessment within this kind of framework challenges traditional designs of instruments to measure learners’ ability to communicate in the language. This requires seeing assessment and the learner’s progress as a process rather than a product. It calls not for a one-time test to measure ability, but rather to study the needs of the students in a given context and their trajectory as language learners. It also includes validating the different linguistic repertoires they bring to the classrooms and provide experiences that will allow them to overcome the challenges they face in and outside of the classroom. As \citet{Larsen-Freeman2017}~explains, “languages are not only acquired or learned, but lived.” \citep{Rosi.Solé2016}.

Assessment is also challenged by contexts in which monolingual ideologies continue to define what goes on in multilingual classrooms. Important pedagogical principles in an ecological approach are the creation of ecologically valid contexts, relationships, agency, motivation and identity. Some guidelines for applying an ecological perspective in language assessment may include (1) notion of “localness” (\citealt{Freeman2000}, \citealt{Tudor2003}), (2) contexts, (3) cultures of learning (\citealt{Tudor2003}; \citealt{CortazziJin1996}), and (4) teaching-learning dynamics \citep{Tudor2001}. In this regard, Silva in this volume calls for including multiple voices and agents in the placement and teaching of Brazilian Heritage Portuguese Community-schools, an ecological pedagogical approach that should be designed after carefully examining the needs of the local community, the teaching approaches in practice and the dynamics between the learner, family and school respectively.

An ecological perspective also defines the classroom and assessment differently. While we have gained much knowledge about the purpose of assessment practices in the last decades, we have also become aware of challenges in applying traditional concepts of assessment to classroom-based assessment~and other communities of practice outside the traditional classroom setting. Classroom assessments as the ones described in the present volume followed \citegen{Turner2012} recommendations when involving the use of strategies by instructors to plan and carry out the collection of multiple types of information concerning student language use, the analysis and interpretation of~data collected for assessment, the feedback received and how the information gathered helps make present and future decisions to enhance teaching and learning (\citeyear[65]{Turner2012}). As some of the chapters in the present volume included examples of classroom assessment techniques used in Higher Education in the US, those assessments must be regarded as unique to a given teaching context and, therefore, approaches to classroom assessment validity need to be dynamic, sociocultural in nature and different depending on the community of speakers and learners.  The multiple alternative assessments proposed in the present volume may fit well under edumetric approaches to validity in assessment as they all aimed to promote and foster good learning behavior and successful progress. Under these parameters, assessment processes become an essential part of everyday classroom practice and involve both instructors and learners in reflection, dialogue and decision making with the ultimate goal of using assessment \textit{FOR} learning (AfL: Assessment Reform Group, \citealt{BroadfootStobart2002,LeungRea-Dickins2007}).

Returning to the analogy posed in the Introduction of the present volume (Chapter 1), we view the dynamics and shaping of language and its related assessment as a continuous fractal formation similar to those fractals that are everywhere in nature. Ecologists have found fractal geometry to be an extremely useful tool for describing ecological systems. Population, community, ecosystems, and landscape ecologists use fractal geometry as a tool to help define and explain the systems in the world around us. The fractal dimension is conceived as a measure of the nature of habitats. In language contact and education settings, the habitats are communities of practice. Different tools are required in population ecology because the resolution or scale with which field data should be gathered is attuned to the study organism (individual learners in educational contexts). Insect movements and~plant root growth follow a continuous dynamic path but the tools required to measure this continuous pathway are very different. Despite multiple shared characteristics of learners' profile, the assessment tools are unique and different depending on the habitat (e.g. classroom walls, community-based learning). In order to avoid habitat fragmentation that produces isolated patches (minority language profiles such as heritage language learners for instance), fractal formation and assessment need to be accessible and inclusive to all in an equitable manner and should keep evolving in varied patterns.

\section{Looking ahead: Future directions for research on alternative assessment in language learning}

\subsection{Challenges of implementing inclusive approaches to assessment in higher education}

One of the strongest influences of assessment in higher education is validity theory (\citealt{PhakitiIsaacs2021,BrownTrace2017}). Both formal and informal assessments need to take into account validity and clear standards in order to elicit data to support learners and to design curriculum. Scholars have argued different dimensions of validity in assessment and the consequences for learners. \citet{McNamaraRoever2006} argued that for the most part, language research investigates technical aspects of validity and not the social dimensions. In other words, assessment is designed for a given context and the impact on this particular community of practice should be considered within construct validity (the extend to which the assessment instrument is intended to measure), content validity (the inclusion of content that is within the scope of the course material covered in class), criterion related validity, and consequential validity (the intended or unintended consequences of assessment for the learners, for instance, being able to graduate or join a program).

Another issue is psychometric test validity when assessing learners in classrooms. As \citet{PhakitiIsaacs2021} argued, “classroom assessment scores cannot be correlated with other external test scores because classroom assessment is used to help the students to improve their skills and overcome any learning difficulties through instructional support” (\citeyear[9]{PhakitiIsaacs2021}). Assessment quality should be the approach. In their model, the authors proposed the following components for what they called \textit{assessment quality} rather than \textit{validity.} This model includes the following components: (1) validity, (2) reliability, (3) practicality, (4) authenticity, (5) ethics, (6) fairness, and (7) effect. It also proposes to start with the intended learning outcomes, and that classroom activities and classroom assessment need to be aligned with the outcomes.

Yet, the challenges classroom instructors continue to face are related to institutional policies as well as the rise in standard tests assessment. This generates in some cases the need for teachers to focus on the tests rather than on the learning process and improvement of their students. Another issue is that alternative forms of assessment require more time spent in the planning stages of the activities. The development of some assessment instruments such as rubrics and group activities to assess different language skills may also require cooperation between teachers. Depending on the class composition, there will be a variety of challenges related to group work and task completion. However, these activities are more aligned with providing learners with opportunities to engage with their context. Awareness and understanding of assessment practices, institutional policies, and students’ academic needs as well as their sociocultural context should be considered in the design of assessment that fosters students' learning through assessment quality criteria.

\subsection{Computer-mediated communication and assessment in multilingual contexts}

  The role of technology in assessment has evolved over the last two decades. The language learner of the 21st Century is constantly learning through different computer-mediated communication outlets. Social media technology and our ability to text and communicate in different languages has revolutionized and expanded the way we communicate and use languages (\citealt{ThorneJakonen2021}; \citealt{Crystal2009}; \citealt{Thorne2008}) and how we assess language learners through the use of videogames (\citealt{ThorneLu2012}; \citealt{Gee2014}). As global citizens, we have moved quickly from texting with limited characters and communicating a message through short texts, to smart phones and social media in which we can basically generate entire documents to share with a wide audience through multimodal approaches to literacy. This has also facilitated processing information in different ways and through multiple languages including the use of technology and \textit{netspeak} \citep{Crystal2009}. These technologies have challenged our reality and altered how we communicate in real time with others around the world. Learners are now more interested in learning foreign languages due to their particular interests. Watching \textit{anime} cartoons, reading comic books such as \textit{Manga} for Japanese learning, and playing videogames, are examples of the motivations for younger generations to use language. Assessment practices are also challenged by these new ways of learning. As such, in multilingual contexts learners are tested in two or more languages simultaneously and their performance is assessed on how well they complete a given task instead of how well they validate and use one of their languages. In this regard, \citet{Larsen-Freeman2018} fittingly points out that “while this type of assessment may not be widely adopted any time soon, computer adaptive testing may well lend itself to more developmentally sensitive, self-referenced assessment, instead of approaches that resemble traditional standardized exams” (\citeyear[63]{Larsen-Freeman2018}).

Recent events such as the pandemic and natural disasters are forcing language educators to rethink assessment during these challenging times. Alternative assessments become the way in which the world continues to collaborate through synchronous and asynchronous instruction. This has also generated new opportunities to engage with learners in different ways that were not previously considered by more traditional assessment methods.

The present volume aimed to provide a snapshot of some alternative assessments that address different realities and needs and initiate a dialogue on future research and additional modalities of alternative assessments in language learning in different communities of practice with other learner profiles. The limited scope of the research findings covered in this volume was due to the contexts the authors were operating mostly in Higher Education in the United States, and the Amazon of Peru and the languages involved (e.g. Heritage language learners of Spanish, Portuguese). Further research in language assessment should include additional language profiles in multiple communities of practice such as sign language learners, heritage language learners of other languages different from Spanish in the U.S. for instance, and study of language assessment in indigenous communities around the globe, to identify a few. Another area of much needed research would be assessment of language learning in hybrid and fully online educational settings. Additionally, pedagogical and assessment challenges (and solutions) remain to be fully explored when addressing disparities between formal and informal assessment practices, validity issues and teaching training. An interesting line of inquiry worth pursuing when advancing our knowledge on alternative assessments for language development would be to determine learners' (and also instructors') dispositions by assessing how receptive they are to the proposed assessments, how willing they are to continue to learn about them, apply them, and be influenced by them. The research findings presented in this volume, though yielding more questions than answers, provides a promising research agenda and dialogue for scholars interested in assessment of language learning.


\sloppy
\printbibliography[heading=subbibliography,notkeyword=this]

\end{document}
